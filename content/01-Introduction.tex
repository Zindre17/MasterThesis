\section{Introduction}

\todo{Denne delen mangler litt, og forsknings-spørsmålene må kanskje endres litt, men ellers klar til gjennomlesing}

This project is a continuation of a project I did in the previous semester. The topic of that article, and therefore also the topic of this article, is using cross reality, or XR, for learning. This is a very broad topic and with help from the supervisor, it was narrowed down. It was agreed upon to focus on virtual reality, or VR, as the technology within the XR domain, and as for what to teach or for users to learn, it was decided to do courses within the computer science field. More specifically Algorithms and computer structures, as this is a course many computer science students struggle with. The mission for the previous project was to make a prototype of a VR application where the user can learn some of the syllabus from the course Algorithms and computer structures. It would do this by attempting to better visualize the abstract parts compared to illustrations in a text book and drawings on a whiteboard or blackboard. The mission for this project, is expanding and improving the prototype developed in the other project and perform tests on it to see if the improvements were adequate. 

The previous project was based on two similar projects from the preceding year under the same supervisor. It used these for inspiration, and to learn from their successes and failures. However, the prototype in this previous project was not based on either of the prototypes from the earlier projects, it was made entirely from scratch. 

\todo{Own section for this?}
The structure of this article is as follows. It will start by, in this chapter, presenting the motivation for this project as well as the research questions developed, and the contributions to the domain this project had or might have. The next chapter is the background. This chapter will present some theory about learning, learning combined with technology, and games. It will also present the technologies used in this project and the software considered and used. Method, the third chapter, will present the preparation/planning and the \todo{not quite sure of this...}implementation of the prototype. The fourth chapter, Results, will describe how the prototype turned out, show some insight into how the prototype was made, and reveal the results of the testing. After that you can read about my thoughts around these results, and ideas for how things could have been done differently or what can be done in the future, in the fifth chapter, Discussion. In the final chapter, six, the conclusion is presented.
%The same supervisor had previously also had two groups of students with similar projects, and this project attempts to build upon and get inspiration and ideas from their successes and failures. The goal was to make something similar to what they made, but improved and with more content. 

%This article is an extension of the work done in my research project last semester. That project's goal was to develop a prototype of a VR application for teaching, and test this prototype. Together with the supervisor it was agreed upon to focus on courses within the computer science studies, and Algorithms and Data structures, and Machine learning were the specific courses discussed. It landed on Algorithms to start with, and maybe add some from machine learning/AI afterwards.

%VR technology has in recent times gotten a lot better in terms of hardware and availability. However, the arena is still not very explored in terms of how to utilize this technology. The mission for this project is to attempt to create a prototype of a virtual environment where learning can happen. 

\subsection{Motivation}
My motivation for this project was rooted in two things. The first was that I had not long before the previous project started tried playing a few games in VR at a friend's place, and I was really fascinated by how immersive the experience could be. When learning it helps, at least for me, when I am immersed and focused on the topic at hand, and I believe VR could be an excellent medium for aiding when it come to learning and teaching. The second was that I was curious about how development for this technology worked. This was more of a motivation for the previous project, as I now know how this happens. The motivation then naturally shifted towards improving upon the prototype for this project. Seeing improvement and growth in one's creations is fun and motivating. 

I imagine the University's motivation for having this topic available for a masters degree, lies in the fact that they want to explore different use cases for new technology, and especially to see if it can be used as aids for their courses in the future. Not that VR is especially new, but it is new in the sense that there has been improvements in the hardware and the availability of this hardware in recent times. This has put VR more in the spotlight, and there is a lot happening in this field currently. The University wants to be a part of this.

\subsection{Research Questions}
The questiones this project will attempt to find an answer to are listed below.
\begin{itemize}
    \item Will the developed prototype indicate that using VR might be suited for teaching Algorithms and Data Structures?
    \item Is VR suited in a school or learning setting?
    \item Is VR suited for teaching Algorithms and data structures specifically?
    \item Is the new prototype improved compared to the previous version?
\end{itemize}

 
\subsection{Contributions}

 