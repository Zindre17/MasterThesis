\section{Discussion}
This chapter will process the information and experience gathered and gained throughout the development and testing, and reflect around them. What was done well? What could have been done better? What have I learned? Does the results make sense? Is the data gathrerd sufficient for reaching a conclusion? Questions like these are what you might find in this chapter. Starting with the process and planning, then moving on to the development, and lastly the testing.

\subsection{Planning and Process}
It would seem as though the previous article was completely forgotten as I have learned next to nothing from it when it comes to planning and process. I redid the same mistakes which I did in the previous project. Too little upfront planning, and too much focus on the development. However, the biweekly meeting with my supervisor forced me to plan a bit ahead, and it gave me a sense of direction and also some sense of urgency. 

\subsection {Development}

\subsection{Testing}
Testing was supposed to have started in late April or early May, but ended up happening mid May. The reason for this was a mixture of things. I was attending a confirmation of my cousin in Stavanger, and due to the SAS-strike I had to travel earlier than needed, and I got home later than planned. During this period I was not able to work too much on the project. This was late in April. Another reason was that I had a realization when I was showing VR to a friend and she wanted to see what I was making. I realized a bunch of tweaks I could make to improve the usability. And some of these tweaks took a lot longer to implement than I had thought. And since quite a few bugs were found in the last projects testing, I thoroughly tested the application in ways it was not meant to be used. This unveiled many bugs which I had not noticed earlier because I, the creator, know how the application is supposed to work and use it thereafter when testing new features. However, some of these bugs were surprisingly hard to squish, but As the testing went smoothly around this time, it seems it was not in vain. 
\begin{comment}
%diskutér resultatene 
Discussion

Planning/Process
The process in this project was handled a bit unprofessionally. I was more eager to develop rather than plan the development. This is not necessarily a bad thing, but the way I handled it could have been much better. The way I did it was that I got an idea of how i wanted the prototype to be, then start to make it. Eventually I would come to a point where I had something which could have been tested, but something about it was not to my liking and I came up with a new design to implement, instead of testing it and basing the new design on feedback from others. This happened at least two times during the project, and cost me a lot of time, with little benefit as I wasted good opportunities to get good feedback.  

Test Results
Now, let us talk a bit about the results from the usability testing. The tests resulted with a SUS score of 67.5 at average. The individual results, however, show that there is a high variance, with the highest score of 85, and the lowest score of 42.5. The thing to take note of here, I think, is that the test subject which gave the lowest score had not taken the course Algorithms and data structures, or any other computer science related subjects, while the other test subjects had taken that course. If we only include the test subjects which had taken this course, the average rises to 75.83, which is a fairly good score. This suggests that the usability of the prototype is linked to the knowledge of the subject, and this is understandable, but might not be a good thing. Because of this, it also suggests that this prototype will perform poorly if used by it self, and not as a supplement to the course. The goal was to make something which could be used as a supplement, but it would, of course, have been nice if it ended up having been self sustainable as well. However, this is only for the usability and does not say anything about the learning/teaching done through this prototype. To test learning is a much more time consuming endeavor. It needs to be done over a longer period of time, and multiple tests, to check for prior understanding, understanding immediately after, and retained knowledge at some point later. This fell outside of the scope of this project.

Research Questions
The first research question introduced int this article was weather VR can be used to help teach complex subjects, and how. Assuming the first part is true, the second part, the "how", then has infinite possibilities. The produced prototype is an attempt at answering this. Weather it accomplishes it is yet to be known. However, the usability test show that people who know something about the subject, find the prototype usable, and might therefor want to use it. This also answers the second question, of how usable the proposed solution is. But then again, usability does not equal engagement(User experience vs Player experience \cite{lazzaro}). And engagement is what one desires in a learning situation. On the other side, bad usability invokes frustration, and not necessarily the good kind of frustration which can transition into mastery. 
\end{comment}