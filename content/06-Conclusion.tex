\section{Conclusion and Future Work}

We have reached the end. This section will conclude this article as well as give suggestions for further work.

\subsection{Conclusion}
If we zoom out a little bit, it becomes clear that the prototype developed in this project is not really useful yet, in its current form. The syllabus coverage even for the only course within is minimal. For it to become properly useful, it would need a lot more content, and the content would need to be good. And even before that point it is not certain that this prototype will help students learn more or more easily, before it is tested in terms of learning. Therefore, what has been done here in this project is only useful for someone which seeks to develop something similar or perform tests on or with something like this. For those who wishes to develop something similar, they can use the current prototype as a starting point, instead of starting from scratch. This could save them time, but it also kind of locks them in to a similar style, unless they do major modifications to the existing bits. For those wishing to further test VR in a learning setting, this prototype could also be used, saving time in terms of development. Though, I would recommend adding tutorials before doing further tests with this prototype. All in all, it has been a good experience for me personally even if it did not contribute anything to ``the bigger picture''. 

\subsection{Future Work}
There is a lot of things that can be done to improve this prototype and there are a lot of thing that can be done to further evaluate it. A good place to start for the prototype would be to add sound and voice to it, as well as a tutorial for all the topics. This would increase its usability as well as make it more entertaining to use. Possibly improving the effectiveness of the teaching as well, should the theory be correct. Other parts of the prototype which could be worked more on is adding more content. More topics to the current course, or new courses and topics for them. 

For further evaluating the prototype, it could be interesting to see if it aids in the learning process. To do this one could team up with the one responsible for the course, and have some of the students taking the course use the prototype as a part of their course work. Then one could do a survey before the course started, or the specific topic evaluating for, to see how much knowledge the students had before starting. This would have to be done by both the students with the prototype included in their course work as well as those without it, such that the results can be measured against. There should also be two more surveys, one to check the knowledge and understanding right after the playing of the prototype or after the topic, to see what was learned in this period, and finally one at a later point to check for retained learning. Then compare the results of the group with the prototype in their course work against those with the normal course work, and see which group had higher learning. It would also be possible, and maybe even a good idea, to have three groups; normal course work, only prototype, and both.
