\documentclass{article}
\usepackage[utf8]{inputenc}
\usepackage{graphicx}
\usepackage{float}
\usepackage{comment}
\usepackage{hyperref}
\usepackage[parfill]{parskip}
\usepackage{listings}
\usepackage{todonotes}
\hypersetup{
    colorlinks=true,
    linkcolor=blue,
    urlcolor=blue,
    citecolor=red
}
\usepackage{listings}
\usepackage{color}
 
\definecolor{codegreen}{rgb}{0,0.6,0}
\definecolor{codegray}{rgb}{0.5,0.5,0.5}
\definecolor{codepurple}{rgb}{0.58,0,0.82}
\definecolor{backcolour}{rgb}{0.95,0.95,0.92}
 
\lstdefinestyle{mystyle}{
    backgroundcolor=\color{backcolour},   
    commentstyle=\color{codegreen},
    keywordstyle=\color{magenta},
    numberstyle=\tiny\color{codegray},
    stringstyle=\color{codepurple},
    basicstyle=\footnotesize,
    breakatwhitespace=false,         
    breaklines=true,                 
    captionpos=b,                    
    keepspaces=true,                 
    numbers=left,                    
    numbersep=5pt,                  
    showspaces=false,                
    showstringspaces=false,
    showtabs=false,                  
    tabsize=2
}
 
\lstset{style=mystyle}

\title{
{UniVRsity}\\
{\large Norwegian University of Science and Technology}\\
{\includegraphics{images/ntnu-logo.jpg}}
}
\author{Sindre Aarstrand}
\date{Spring 2019}

\begin{document}
\pagenumbering{gobble}
%\maketitle
%\clearpage\mbox{}\clearpage
\begin{abstract}
    Virtual Reality is an ``up and coming'' technology. Hardware became available to consumers in 2016, and several big companies are investing a lot to improve these technologies, as well as making them more affordable. The main focus of VR from the start has been immersive games, but there is belief that there are other areas where VR can be utilized as well. 
    
    This thesis wanted to experiment with using VR in a learning setting. To do this, an application was developed with the aim of helping to teach topics from different university courses to students. The course ``Algorithms and data structures'' was the course chosen for the prototype, as this is a course many struggle with. This application was developed in the Unity 3D game engine, and it was evaluated through performing user tests followed by a questionnaire. The results revealed nothing in regards to learning, but rather how usable the developed prototype was, and whether the participants believed in using this technology for learning. The usability was not great, but it was not particularly bad either, as it could with minor additions become a lot better. And yes, the participants believed this technology should be used in learning. Unanimously so. 
    
    The prototype is hosted publicly on \href{https://github.com/Zindre17/UniVRsity}{Github}\cite{Github} and is available for use, with or without modifications, by anyone interested in doing so. A video tour of the final prototype can be watched by following \href{https://youtu.be/VUHdUxoRH5Y}{this link}\cite{UniVRsity}.
\end{abstract}

\newpage
\pagenumbering{roman}
\tableofcontents
\newpage
\pagenumbering{arabic}
\section{Introduction}

\todo{Denne delen mangler litt, og forsknings-spørsmålene må kanskje endres litt, men ellers klar til gjennomlesing}

This project is a continuation of a project I did in the previous semester. The topic of that article, and therefore also the topic of this article, is using cross reality, or XR, for learning. This is a very broad topic and with help from the supervisor, it was narrowed down. It was agreed upon to focus on virtual reality, or VR, as the technology within the XR domain, and as for what to teach or for users to learn, it was decided to do courses within the computer science field. More specifically Algorithms and computer structures, as this is a course many computer science students struggle with. The mission for the previous project was to make a prototype of a VR application where the user can learn some of the syllabus from the course Algorithms and computer structures. It would do this by attempting to better visualize the abstract parts compared to illustrations in a text book and drawings on a whiteboard or blackboard. The mission for this project, is expanding and improving the prototype developed in the other project and perform tests on it to see if the improvements were adequate. 

The previous project was based on two similar projects from the preceding year under the same supervisor. It used these for inspiration, and to learn from their successes and failures. However, the prototype in this previous project was not based on either of the prototypes from the earlier projects, it was made entirely from scratch. 

\todo{Own section for this?}
The structure of this article is as follows. It will start by, in this chapter, presenting the motivation for this project as well as the research questions developed, and the contributions to the domain this project had or might have. The next chapter is the background. This chapter will present some theory about learning, learning combined with technology, and games. It will also present the technologies used in this project and the software considered and used. Method, the third chapter, will present the preparation/planning and the \todo{not quite sure of this...}implementation of the prototype. The fourth chapter, Results, will describe how the prototype turned out, show some insight into how the prototype was made, and reveal the results of the testing. After that you can read about my thoughts around these results, and ideas for how things could have been done differently or what can be done in the future, in the fifth chapter, Discussion. In the final chapter, six, the conclusion is presented.
%The same supervisor had previously also had two groups of students with similar projects, and this project attempts to build upon and get inspiration and ideas from their successes and failures. The goal was to make something similar to what they made, but improved and with more content. 

%This article is an extension of the work done in my research project last semester. That project's goal was to develop a prototype of a VR application for teaching, and test this prototype. Together with the supervisor it was agreed upon to focus on courses within the computer science studies, and Algorithms and Data structures, and Machine learning were the specific courses discussed. It landed on Algorithms to start with, and maybe add some from machine learning/AI afterwards.

%VR technology has in recent times gotten a lot better in terms of hardware and availability. However, the arena is still not very explored in terms of how to utilize this technology. The mission for this project is to attempt to create a prototype of a virtual environment where learning can happen. 

\subsection{Motivation}
My motivation for this project was rooted in two things. The first was that I had not long before the previous project started tried playing a few games in VR at a friend's place, and I was really fascinated by how immersive the experience could be. When learning it helps, at least for me, when I am immersed and focused on the topic at hand, and I believe VR could be an excellent medium for aiding when it come to learning and teaching. The second was that I was curious about how development for this technology worked. This was more of a motivation for the previous project, as I now know how this happens. The motivation then naturally shifted towards improving upon the prototype for this project. Seeing improvement and growth in one's creations is fun and motivating. 

I imagine the University's motivation for having this topic available for a masters degree, lies in the fact that they want to explore different use cases for new technology, and especially to see if it can be used as aids for their courses in the future. Not that VR is especially new, but it is new in the sense that there has been improvements in the hardware and the availability of this hardware in recent times. This has put VR more in the spotlight, and there is a lot happening in this field currently. The University wants to be a part of this.

\subsection{Research Questions}
The questiones this project will attempt to find an answer to are listed below.
\begin{itemize}
    \item Will the developed prototype indicate that using VR might be suited for teaching Algorithms and Data Structures?
    \item Is VR suited in a school or learning setting?
    \item Is VR suited for teaching Algorithms and data structures specifically?
    \item Is the new prototype improved compared to the previous version?
\end{itemize}

 
\subsection{Contributions}

 
\newpage
% Teori om teknologi, læring, software osv.
\section{Background}

This section will provide background information about the technologies and software utilized in this project, as well as theory addressing learning, learning with multimedia, and enjoyment in games. It will build the foundation for the planning and design choices made in this project. 

\subsection{Virtual Reality}
\todo{add images of second gen}

Virtual reality(VR) has many definitions, but in the context of this project VR refers to a virtual environment with spatial and orientation tracking of a head mounted display(HMD) and controllers. In VR the virtual environment's visuals are 100\% computer graphics, while the interaction with these generated visuals are done through the movement of the physical body of the user. VR is also a part of the term XR, or Cross Reality, which includes VR, mixed reality(MR), augmented reality(AR), and a few more. Most of these use a mixture of the real world and graphics in the visuals, which contrasts to VR's 100\% graphical visuals.

The HTC Vive(figure \ref{fig:htcvive}) and the Oculus Rift(figure \ref{fig:oculus}) are the first generation of consumer VR-kits and are also currently the standard. This first generation of VR devices first hit the consumer market in 2016, but development kits and prototypes have been available a lot longer. The second generation of VR is already announced, some even available for pre-order, and the thing about these seems to be the lack of external sensors for tracking the headset and controllers. They are now integrated in the HMD. This will make transportation and setting up a lot easier. The user experience should also be better as this should remove the dizzying and or nauseating effect when the sensors loose track of the headset. They also boast better visuals in terms of less "screen door"-effect, higher resolutions, better colors, and some even eye-tracking. Both HTC and Oculus has announced tether-less versions for gaming as well. The previous generation's tether-less version were not compatible with games, as they did not support controllers suitable for this purpose.

\begin{figure}[H]
\centering
\includegraphics[width=\textwidth]{images/Htcvive.jpg}
\caption{HTC Vive Pro and controllers}
\label{fig:htcvive}
\end{figure}

\begin{figure}[H]
\centering
\includegraphics[width=\textwidth]{images/oculusrift.png}
\caption{Oculus Rift and Touch controllers}
\label{fig:oculus}
\end{figure}
%The exact origins of virtual reality is uncertain due to the difficulty of defining an alternative existence. Then what is virtual reality? In terms of functionality it is a simulation where the world one sees is computer generated graphics which responds to input from the user in real-time. %[https://books.google.no/books?hl=no&lr=&id=0xWgPZbcz4AC&oi=fnd&pg=PR13&dq=virtual+reality+origin&ots=LDgAgW0Ncw&sig=gG7pQ1B9dXtZTtHGqBuIAyWE3uw&redir_esc=y#v=onepage&q=virtual%20reality%20origin&f=false ].
%Virtual reality environments exist in several different forms. Some use projectors and rooms, while others use head mounted displays, or HMDs, and controllers. In recent times, HMDs have seen a surge in popularity and interest, most likely due to the increase in hardware capacity and decrease in size and price. Within this group or form of VR, there are kind of two categories. Due to the fact that technology has evolved to where it is now, even some smart phones are able to power light virtual reality experiences. This is the first category which uses regular smart phones or dedicated standalone headsets with equivalent specifications to power the virtual reality experience. These are experiences are limited in functionality and power due to the size. The second category are headsets and controllers which are powered by an external computer. This allows for far more computationally heavy applications. I will develop for the second category, since this allows the use of controllers and more complex interaction for better immersion. 

\subsection{Software}

Unity and Unreal Engine 4 were the only game engines considered for developing this application. They are the most common engines for independent developers to chose, especially for VR development. Both are also free to use for non-commercial use, and mostly free for commercial use as well, with a few limitations and or exceptions. This makes these engines ideal for beginners just starting out or wanting to test game development. One of the major differences in these engines is the lighting. In Unity the lighting is baked, while in Unreal it is not. This gives Unreal the edge when realistic graphics is wanted. Unreal also focuses more on 3D games giving them the advantage there. On the other hand, this gives Unity the edge in 2D games. They also use different programming languages.
%The biggest difference in these engines is the lighting. In Unity the lighting is baked, while it is not in Unreal, which gives Unreal a more realistic look by default. The trade-off for better more realistic lighting is that it is more work to develop in this engine? %todo check if this is the case
\subsubsection{Unity}
Unity has support for several languages, but the two most used are C\# and Javascript. To such an extent that Unity is dropping/has dropped support for other languages. Unity is based around game objects and scripts, or components, attached to these game objects. These components, if they inherit from the default unity class: MonoBehaviour, have a start function which is run at the time the object is created, and an update function which is run every frame. MonoBehavior also has access to many more functions, but these are the basics. Unity is a mixture of visually programming and coding. If a variable in Unity is marked as public, it can be accessed visually within the editor. Unity also has an Asset store within the editor where all sorts of assets, like models, textures, sounds, and visual effects, can be purchased or downloaded for free. These assets are made by the community, and the quality is therefore varying, but there is a rating system to help chose wisely.

\begin{figure}[H]
\centering
\includegraphics[width=\textwidth]{images/unit3d.png}
\caption{Unity logo}
\label{fig:unity}
\end{figure}

\subsubsection{Unreal Engine}
Unreal Engine uses C++ for development. However, it also provides "blueprints" which is an interface for scripting visually(drag n' drop). Strictly speaking, this means that both Unreal Engine and Unity can create games without having to code. Unreal is based around actors, which are almost equivalent of game objects in unity. Actors have components just like game objects in unity has. In addition they can have blueprints or scripts attached to them.  %todo lære litt unreal for å snakke litt om det

\begin{figure}[H]
\centering
\includegraphics[width=\textwidth]{images/unreal.png}
\caption{Unreal Engine logo}
\label{fig:unreal}
\end{figure}

\subsubsection{SteamVR plugin}
Since there exist many different forms of controllers and HMDs, Valve has created a plugin for Unity and Unreal Engine, SteamVR plugin\cite{SteamVR_Plugin}. This plugin makes it possible to support a wide variety of devices using the same code. Using this will minimize the code needed to support a wide variety of devices. The difference between a VR game and a normal 3D game is that VR games need to somehow reflect the users movements and actions in real life within the game. I.g move the hands to where the user moves their controllers, look in the direction the user is pointing their head, and all these things. The focus of this plugin is therefore to make the camera follow the users head, load models of the controllers in use, make these controller models follow the users hands, handle the inputs of these, and estimate how the users hands look while using these controllers.



%Teaching

%Regular classroom teaching has its flaws and advantages. One of the biggest flaws is probably that it tries to teach a whole group at the same time. Because the teacher - student relation is one(few) to many, the entire group will necessarily be taught with the same pace. This is where classical method of teaching fails. Everyone learns at their own pace, and therefore some will get “left behind” while others will get bored. %A virtual single player game can fix this issue by allowing each user to progress at their own pace by enabling functionality such as skip or rewind. However, this comes at the cost of eliminating communication.  

%Learning is enhanced when the learner is more immersed. This is quite obvious since learning requires focus, and immersion is, in a sense, a measurement of focus. A student who is more eager to answer or ask questions, or active, is a more immersed student, and is therefore more likely to learn more. Computers have gradually and increasingly been introduced as a supplement in the traditional classroom teaching method in recent years, as an attempt to improve teaching. Even though a computer requires interaction, which sort of forces at least a certain level of focus, it is easy to get distracted by what is happening around you. The belief here is that VR will be far superior to an ordinary computer screen in inducing immersion, since wearing a VR-headset blocks off your vision and partially your hearing.   



\subsection{Multimedia Learning Theories}

The cognitive theory of multimedia\cite{Mayer2014}, which is based upon the cognitive load theory\cite{CognitiveLoadTheory}, states that the visual input and the auditory input are processed on different channels, with a limited cognitive capacity for processing these inputs. This means if the visual channel is overloaded with input, the auditory channels processing will suffer, and vice versa. From this follows that irrelevant images, sounds, or words can harm the learning in the learner, and should therefor be avoided. In the book Multimedia learning\cite{Mayer2009}, Mayers present twelve principles for better design when using multimedia for presenting material. These principles are backed by both the cognitive theory of multimedia, and empirical studies.

The principles are as follows: 
\begin{itemize}
    \item Coherence Principle - People learn better when extraneous words, pictures and sounds are excluded rather than included.
    \item Signaling Principle - People learn better when cues that highlight the organization of the essential material are added. 
    \item Redundancy Principle - People learn better form graphics and narration than from graphics, narration and on-screen text.
    \item Spatial Contiguity Principle - People learn better when corresponding words and picture are presented near rather than far from each other on the page or screen.
    \item Temporal Contiguity Principle - People learn better when corresponding words and pictures are presented simultaneously rather than successively. 
    \item Segmenting Principle - People learn better from a multimedia lesson is presented in user-paced segments rather than as a continuous unit.
    \item Pre-training Principle - People learn better from a multimedia lesson when they know the names and characteristics of the main concepts.
    \item Modality Principle - People learn better from graphics and narrations than from animation and on-screen text.
    \item Multimedia Principle - People learn better from words and pictures than from words alone.
    \item Personalization Principle - People learn better from multimedia lessons when words are in conversational style rather than formal style.
    \item Voice Principle - People learn better when the narration in multimedia lessons is spoken in a friendly human voice rather than a machine voice.
    \item Image Principle - People do not learn better from a multimedia lesson when the speaker's image is added to the screen.
\end{itemize}

%All these principles are important to consider, and the ones I deemed the most important are the coherence principle, the redundancy principle, the modality principle, the personalization principle, and the voice principle. 
The fact that VR gives the opportunity to do anything or be anywhere, has the potential to unintentionally be very distracting if the environment is very detailed  and or dynamic, as this can draw the attention away from the presented material the user is supposed to interact with and learn from. However, always keeping the coherence principle in mind, and thinking twice when adding content of whether it is an aid in the learning process or not, can help in avoiding such a situation. An interesting environment will probably pique a users interest initially but might be distracting and decrease the efficiency and effectiveness of the learning material. There seems to be a trade-off there, unless the environment changes into something less distracting when the application requires the users attention elsewhere. 
%According to the coherence principle, people learn more or better if there are no irrelevant information or distractions. This can be anything from sound effects or animations in power point slide transitions, to digressions, to irrelevant imagery. I think this point in particular is important to take note of. VR gives the opportunity to do anything or be anywhere, which means it could be easy to fall for the temptation of creating a really interesting environment. This will probably pique the users interest, but might end up harming the application in terms of learning, if it distracts attentions away from the material presented.

When it comes to deciding how to design the instructions, the redundancy principle suggests it has graphics and narration, but without text. The reasoning lies with the redundant text prompting extra visual processing, resulting in less learned. And while in the area, the modality principle suggests that instructions should consist of graphics and voice, rather than graphics and text. By having both text and graphics the visual channel might get overloaded, while with graphics and voice, the input is split among the auditory and visual channels.

Taking the pre-training principle into consideration, it would suggest that having this prototype as an addition to lectures or other classic teaching methods, rather than as a stand-alone product, is more effective. The reason for this is that knowing some of the terms and characteristics of a concept in advance means that the user has kind of allocated a space in his/her knowledge space where the new information can go. An analogy might be a cake. A cake can be eaten without knowing the ingredients or how it is made. However, presented with only the ingredients and it is very hard to imagine what they become when combined in a specific way.

One of the principals that the masters degree project of Kong and Kruke\cite{KongOgKruke} followed well, is the segmentation principle. The prototype allows the players to control the pace of the learning to a high degree. The players can choose which sorting algorithm to learn, and they can do it step by step with as much time as they need per step. This is something to keep in mind for this project too. 
%Since the goal of this project is not to replace the existing teaching method, but be an addition to it. The pre-training principle says that this is advantageous, as a learner who already knows the names and characteristics of the main concepts will learn more deeply from a multimedia instruction. If we take a look at the masters degree report of Kong and Kruke\cite{KongOgKruke}, they organized their game into single algorithms which enables the player to control the pacing of the information, to a certain degree. This is what the segmentation principle suggests, and is therefore to be considered in this project as well. 

Lastly from these principles it is worth noting that people learn better when words are in conversational style rather than formal style, because the learner will be prompted to try harder when they feel that someone is talking to them specifically. Also, if these words are in the form of voice rather than text, the learner will be more engaged if the voice is human and kind compared to robotic and cold. This is according to the personalization and voice principles. 

\subsection{Video Games and Fun}

Even though this project is based around an educational game, it is still a game, and games are inherently made to be fun or entertaining. This is something which should, therefore, also be considered in this project. In 2005, Penelope Sweetster and Peta Wyeth, made an attempt at creating a model for evaluating and designing fun games. This model was based on Csikszentmihalyi's Flow theory\cite{Flow}, which is a theory or model on enjoyment in general, and the adapted version was named, creatively so, GameFlow\cite{GameFlow}. 

GameFlow consists of eight elements; Concentration, Challenge, Player Skills, Control, Clear Goals, Feedback, Immersion, and Social Interaction. For someone to enjoy a game, all of these elements does not necessarily need to be fulfilled, but at least a few of them. As an example, a game can be fun even though it does not include social interaction. Anyways, let's review what lies within these elements. 

\begin{itemize}
    \item Concentration - Games should require concentration, but also help the player in doing so
    \item Challenge - Games should be sufficiently challenging
    \item Player skills - Games should support player skill development and mastery
    \item Control - Players should feel that they are in control of their actions in the game.
    \item Clear goals - Games should provide players with clear goals at appropriate times
    \item Feedback - Players should receive appropriate feedback at appropriate times
    \item Immersion - Players should experience deep but effortless involvement in the game
    \item Social Interaction - Games should support and create opportunities for social interaction
\end{itemize}

These eight points might prove helpful when making design choices for the prototype

\todo{add lazzaro}
\begin{comment}
- cognitive theory of multimedia
- cognitive load theory

- Adding stuff like visual effects, sounds and animations, will "use up" some of the cognitive capacity of the learner on , resulting in less capacity available for learning the material. 

- multimedia learning == learning from words and pictures

- previous studies have found that immersive VR, when not carefully designed, can distract the learner with animations and dynamic environments, from where the focus should be. E.g. spoken words or a demonstration. 

- coherence principle and segmenting principle

- Coherence: people learn better when unnecessary words, sounds, and  pictures are excluded. These divert the attention away from the important stuff, disrupt the process of organizing the information, and may lead to improper integration of new information with existing knowledge in the learner.(Mayer 2009). Research which backs up this theory showed that instructions in desktop VR in 2D was superior to desktop VR in 3D, due to the fact that the 3D environment requires more cognitive processing. Also research has found that cartoon-like graphics outperform photo realism, due to the same fact. 

- Segmenting: people learn better when the lesson is presented in user-directed segments, and not as a continuous unit. This boils down to letting the learner decide the pacing of incoming new information. VR animations are usually continuous and therefore not abiding to this principle, to some extent. 
- These principles shows that VR is very susceptible to not being as effective for teaching. 


- Give the students something to do, not something to learn(learning by doing) - john dewey

- Low-knowledge level learns best from images and speach(with instructions), while High-knowledge level learns best from images(without instructions) alone. 

- Gameflow

\end{comment}
\newpage
%denne delen skal forklare hvordan spillet har blitt implementert og hvordan teorien har blitt utnyttet
\section{Methodology}

This section will present how the prototype was planned, the different choices made in regards to how the development should be done, and also how the testing should be performed. 

\subsection{General Plan}
The plan going into this project was to continue the development of the prototype developed in the previous project. Making improvements and adding new features to it, and then test it again to see if it was actually improved. As far as design goes there will not be any major changes, unless something which forces this occurs. There will however be additions to it, and these additions will try to achieve a similar style. This is to be consistent with how things work, not causing unnecessary confusion in the players. The planning of the expansion of content and making improvements will happen in iterations. Come up with a next step(a general idea, no specifics), figure out how to implement this idea, implement it, then from there figure out where to go next. The next step for each iteration would mostly come from the meetings with the supervisor.  

Coming from the previous project the structure of the prototype was just a single scene. However, the plan for a finished product looked something like that which can be seen in  \autoref{fig:plan}. The question marks represent any number of courses or topics, and the structure is hierarchical. The overworld is at the top of the hierarchy and form there the player can navigate to any course hub. The course hubs are the next level of the hierarchy where the player can either navigate back to the overworld, or to any topic within the course. The topics are the lowest level in the hierarchy, and the player can only navigate back to its course hub from there. This hierarchy will be the guide when adding new content in this project.


\begin{figure}[H]
\centering
\includegraphics[width=\textwidth]{images/Plan.png}
\caption{The idea of how the scenes in the final product should be arranged. The question marks can be any number of boxes.}
\label{fig:plan}
\end{figure}



\subsubsection{Iteration 1}
The goal for the first iteration was to add another topic from the course Algorithms and data structures. The topic chosen was data structures, despite the previous project having some ideas around tree traversing algorithms. The reason for this, was to better represent the content of the course. It is not only about algorithms. Another idea which were considered was to start adding another course like Machine learning. This would imply a scene structure like the one seen in \autoref{fig:plan_potential}. However, this was discarded for the time being, and the structure would instead look like the one presented in \autoref{fig:plan_final}
\begin{figure}[H]
\centering
\includegraphics[width=\textwidth]{images/potential_further_dev.png}
\caption{How the scenes in the game would be arranged, if the machine learning course was added instead of the data structure topic.}
\label{fig:plan_potential}
\end{figure}

\begin{figure}[H]
\centering
\includegraphics[width=\textwidth]{images/actual_result.png}
\caption{The plan for how the scenes in the game would be arranged, after the first iteration}
\label{fig:plan_final}
\end{figure}

\subsubsection{Iteration 2}
The next iteration after the initial data structure scene was added, was to go back to the sorting algorithm scene and add a better and more commonly used algorithm to the mix. The algorithm chosen was quick sort, as it fits these criteria.

\subsubsection{Iteration 3}
In the third iteration, the idea was to add some form of use case to each of the data structures. The idea behind that idea was to show the player how this data structure could be used in a real life scenario. As the data structures added in the previous iteration was a stack and a queue, and the supervisor teaching visual computations the decision for use case landed on an image region growing algorithm. This algorithm need to keep track of which pixels to explore, which works with both a stack and a queue, but will behave different in how the pattern grows.

\subsubsection{Iteration 4}
Seeing as the overworld, course hubs, and the navigation between them was still not implemented, it was time to do this. This was planned already back in the previous project but not achieved in the duration of it, and it was long overdue to get started on this. It would still be arranged like the figure from iteration 1,  \autoref{fig:plan_final}. My supervisor had an idea of making the overworld look similar to a school ground or university ground. However, due to time constraints, the looks of it was not the priority. The priority was getting the functionality down. Considering that I have no experience in creating 3D models, textures, or other forms of visuals, the results would probably be sub-par and it would cost me a lot of time. 

\subsubsection{Iteration 5}
After the overworld and navigation was added, I wanted to add the merge sort algorithm. Though this required a lot of rework to the sorting scene in the state it was at, at the time. Merge sort is different to the other three sorting algorithms in that it creates new arrays in the process.  

\subsubsection{Iteration 6} \label{sec:It6}
This time I had realized that I had forgotten to improve things which were commented on or discovered during the testing in the previous project. I had been too focused on adding new features and improving stability and architecture, and forgotten probably the most important part; improving the usability. This iteration would therefore consist of making a lot of improvements in terms of usability and thoroughly testing the application to find and squash bugs before the testing started. 

\subsection{Development}
The choice of game engine and programming language fell on the decision from the previous project, namely Unity and C\#, as this project's intentions are to directly improve upon this same prototype. Recreating the previous prototype in another game engine would be a lot of unnecessary work, since the plan is to keep everything and just build on top of the existing prototype, and the other game engine would need to be learned, and probably the programming language the game engine uses as well. This does not fit in with the time schedule of this project. In addition, the gained experience with the game engine from the previous project would shorten further development times. In other words, I was already familiar with using Unity and C\# at this point and did not really need to spend more time on learning the engine. This project would continue to work on the same public repository on Github as in the previous project. 

This project does not intend to alter the architecture of the existing prototype. It should still focus on being modular, as this makes it easier for other developers to extend it and improve upon it. Having the prototype keep evolving and eventually encompassing a plethora of courses is kind of the idea or hope here. However, considering the time constraint, making things work is more important than having a good architecture. At least in the short term scenario. 

\subsubsection{Requirements}
Following is a list of functional requirements for the project, sorted by the scenes. Non-functional requirements were not set, except for keeping the prototype performance efficient.

\subsubsection*{Common to all scenes}
\begin{itemize}
    \item Player - A camera and controller reflecting the physical orientation and position of the player hand his/her hands/controllers. Also has a laser pointer in the right hand for interacting with the world
    \item Show outline on interactive objects when hovering over with the laser pointer
    \item Interact with interactive objects by pressing the trigger button on the controller with the laser pointer while pointing at it
    \item Buttons - Interactive object
    \item Show which buttons are disabled by changing their color, and the color of the outline when hover over
\end{itemize}

\subsubsection*{Over world}
\begin{itemize}
    \item Show available courses as different buttons
    \item Course buttons - Loads the button's course
    \begin{itemize}
        \item Algorithms and Data Structures
    \end{itemize}
\end{itemize}

\subsubsection*{Algorithms and Data Structures}
\textbf{Course hub(entry point):}
\begin{itemize}
    \item Show available topics as buttons
    \item \textbf{Topic buttons} - Loads the button's topic
    \begin{itemize}
        \item Data structures
        \item Sorting algorithms
    \end{itemize}
    \item Back navigation button - Go back to the over world
\end{itemize}

\textbf{Data structures:}
\begin{itemize}
    \item \textbf{Play mode} - Learn how the selected structure works
    \begin{itemize}
        \item Show available structures as buttons
        \item \textbf{Structure buttons} - Go to play mode for selected structure
        \begin{itemize}
            \item Stack
            \item Queue
        \end{itemize}
        \item \textbf{Stack and Queue} - Same play mode
        \begin{itemize}
            \item Show a representation of the data structure
            \item Push/Enqueue button - Animate how the structure handles addition of data
            \item Pop/Dequeue button - Animate how the structure handles removal/retrieval of data
            \item Use case button - Go to use case mode for this structure
            \item Back navigation button - Go back to course hub
            \item Show explanation of each button
            \item Show error message when doing an illegal action containing the error (Overflow/Underflow) and when it occurs
        \end{itemize}
    \end{itemize}
    \item \textbf{Use case mode} - See how the selected structure can be used in a real life scenario
    \begin{itemize}
        \item \textbf{Stack and Queue} - Same use case: Image region growing
        \begin{itemize}
            \item Back navigation button - Go back to play mode for this structure
            \item Show a description of what this use case is when entering the use case mode. Press the button on this panel to hide it and start the algorithm.
            \item Show a black and white image with random pattern. Each pixel is an interactive object.
            \item Pixel(interactive object) - Select (show a border color to indicate that it is selected)
            \item Show a representation of the data structure selected before entering use case mode
            \item Show a representation of the the item from the data structure is currently being read/used ("Next" as it points to the next pixel to check)
            \item Show a representation of the internal state of the algorithm (Pattern so far, visited pixels, which pixel is next, etc...)
            \item Show an explanation of the color coding in the representation of the algorithm
            \item \textbf{Action buttons}
            \begin{itemize}
                \item Push/Enqueue - Animate addition of the selected Pixel to the data structure
                \item Pop/Dequeue - Animate removal of next pixel from the data structure and into the "Next" position. 
                \item Check - Animate the checking(whether it is or is not part of the pattern) of the pixel.
            \end{itemize}
            \item \textbf{Algorithm interaction buttons}
            \begin{itemize}
                \item Demo(toggle button) - Start/stop the demonstration of the algorithm. The demonstration animates how the algorithm works step by step.
                \item Prev - Undo the last step in the algorithm(animate)
                \item Next - Do the next step in the algorithm(animate)
            \end{itemize}
            \item An attempt at doing an action the algorithm does not expect, it will not perform the action and instead hint at the expected action by blinking the required elements and action button a few times. 
            \item Data items popped or dequeued into "Next" will fire a laser at the pixel it represents. To indicate which pixel should be checked next
        \end{itemize}{}
    \end{itemize}
\end{itemize}

\textbf{Sorting Algorithms}
\begin{itemize}
    \item Back navigation button - Go back to course hub
    \item Show available sorting algorithms as buttons
    \item Show the array as a horizontal set of cuboids(array elements) on top of another cuboid(array stand or representation of the memory location). The elements has a height relative to their value: high value means a tall cuboid, and a low value means a short cuboid.  
    \item Both the array and the the elements are interactive
    \item Show a Cube representing the storage (where array elements can be temporarily stored)
    \item Show a representation of the internal state of the algorithm
    \item Show pseudo code for the selected algorithm
    \item \textbf{Sorting algorithm buttons} - Generates a new random array and changes to this algorithm
    \begin{itemize}
        \item Bubble sort
        \item Insertion sort
        \item Quick sort
        \item Merge sort
    \end{itemize}
    \item \textbf{Algorithm interaction buttons}
    \begin{itemize}
        \item Demo(toggle button) - Start/stop demonstrating the algorithm. The demonstration shows the algorithm step by step in the correct order and explains what it is doing.
        \item Prev - Undo the last step in the algorithm(animate)
        \item Next - Do the next step in the algorithm(animate)
        \item New array - Generate a new random array
        \item Restart - Resets the algorithm, and array back to start    
    \end{itemize}
    \item \textbf{Action buttons}
    \begin{itemize}
        \item Compare - Compare the value of two elements
        \item Swap - Swap the values/position of one or two element(s)
        \item Store - Clone one of the elements in the array to the storage
        \item Copy to - Two step action: Copy selected element value into the next selection (limitations: can not copy to storage, use Store instead)
        \item Pivot - Sets the selected element as a pivot
        \item Split - Splits the selected array in two halves
        \item Merge - Start the merge of two arrays
    \end{itemize}
    \item An attempt at doing an action the algorithm does not expect, it will not perform the action and instead hint at the expected action by blinking the required elements and action button a few times. 
    
\end{itemize}

\subsubsection{Implementation}
This part will talk a bit about how the major parts in the game was coded, how they work, and how they interact with each other.

\subsection{Evaluation}
The evaluation in this project will be fairly similar to the form of evaluation performed in the previous project, whose form was briefly touched upon in the Previous Work section.  This time there will be a short explanation before the evaluation starts about what the game is, and its intentions. It will give the participants some context in the form of what they can expect in terms of genre and setting. After the introduction of the game, they will play the game. No one will give guidance to them unless they explicitly ask for help. If a participant asks for help, it is a sign that some part of the game is not intuitive or something should be better explained within the game. The participants will be informed of this, to increase the threshold for asking for help. After the participants feel they have done enough, they will be asked to fill out a form. This form will contain a section regarding some background information of the participant, a part with the standard SUS questions for getting a usability score, and some general feedback regarding the game and their experience with it. The targets for this evaluation will be people who has taken or is taking the Algorithms and data structures course, as well as people who has not taken the course but has some experience and knowledge of programming. The level of VR experience a participant has is not of much importance, though a varied spectrum could be nice. This would open up the possibility to check for a connection between previous experience with VR and their understanding of the prototype. As already mentioned the previous project was only able to perform the testing with five people. This time around, the aim is to at least perform tests with double the amount of people.

\newpage
\section{Results} 
\label{sec:Result}
\todo{Denne delen trenger en del omskriving/omstrukturering og legge til test resultatene} 

This section will present the results, in regards to development and testing. The part about development will touch upon structure, specifics of some components, and some bugs and stumbles which are worth mentioning. The testing part will present test results without any interpretation, just numbers. The interpretation of the test results will come in the next section: Discussion. 

\subsection{The Prototype}
The development was halted in mid May, 2019, to give time for testing and writing the report. The final version of the prototype can be seen in \href{https://youtu.be/yOpMw9f9AtY}{this} video\cite{UniVRsity}. It shows how the prototype works, how it looks, and a general look into all its content. Some of the most important pieces of this software and how they work will be explained right after this, and the different pieces will be classified as either one of these: 
\begin{itemize}
    \item A component - A C\# class inheriting from MonoBehaviour.
    \item A prefab - A template for a game object which can be used to spawn predefined game objects at run-time or for speeding up the scene building process.
    \item A class - A C\# class.
    \item An interface - A C\# interface.
\end{itemize}
For more details the code can be found on \href{https://github.com/Zindre17/UniVRsity}{Github} Then finally an exhaustive list of the features in the prototype will be presented. 



\subsubsection{Overworld}
The overworld is just a plain with a wall on top of it, and some hovering text, showing the name of the prototype; ``UniVRsity''. The wall contains buttons for all the courses in the prototype, which is at this point in time only one; ``Algorithms and data structures''. 
%This scene, as well as all other scenes contain a ``\_FallbackGame'' prefab. This prefab is very shorty described the player and everything all scenes need, or the core. It is explained in more detail in \autoref{sec:majorComponents}. The only other ``exciting'' parts are the ``MenuButton'' component attached to the buttons, and the ``MainSceneChanger'' component.
The interesting pieces in this scene are as following:
%\textbf{``\_FallbackGame''}
\begin{itemize}

    \item \textbf{LaserPointerPlayer} 
    \begin{itemize}
        \item[] The ``LaserPointerPlayer'' prefab was a modified version of the ``Player'' prefab which is shipped out of the box with Steam VR. The ``Player'' prefab controls the tracking of the controllers and the HMD. The modified version removed the hand models, and their gesture simulation as this is not needed for the interaction within the game. Instead the hand models were replaced by spheres and attached to the right hand was a laser pointer which is used for interaction. Along with this laser pointer model, a component(``LaserPointer'') was added to control this model. 
    \end{itemize}
    
%fallback
    \item \textbf{\_FallbackGame}
    \begin{itemize}
        \item[] ``\_FallbackGame'' is a prefab, and it came to be after a bug which was found late in the development. Shortly explained, the ``LaserPointerPlayer'' prefab object would not be destroyed when changing scenes and there would be multiple of these when more than one scene was used. More on this in the next section. The purpose of this ``\_FallbackGame'' object was to check for a player(a ``LaserPointerPlayer'' prefab) and if there was none, instantiate one, else, destroy itself. This made it possible to start the game from any scene and still visit other scenes without getting multiple players, and without only having a player in one of the scenes. This was the initial function of this prefab, but since there would only ever be one of these objects, the opportunity arose to move all other components which were required by all scenes into this same object. Kind of making it into the core of the game, and hence the name ``\_FallbackGame'', where ``Fallback'' comes from the part where it checks whether or not it is needed, and ``Game'' coming from the contents being all the stuff required everywhere within the game. The contents are, excluding the player, the ``EventManager'' and the ``ColorManager'' components, as of the time of writing. 
    \end{itemize}
    
    
%eventmanager
    \item \textbf{EventManager}
    \begin{itemize}
        \item[] The ``EventManager'' is a component which contains every event which could happen in the game, and allows other classes or components to subscribe to these events. The events are set up with delegate types for the different types of events, and matching static events using these delegate types. Having the events be static means that other components does not need a reference to the active ``EventManager'' in order to subscribe to events. Then, all a component need to do in order to subscribe to an event is to define a function which takes the same parameters as the delegate type of the event, and append that function to the event. Like this: 
        \begin{lstlisting}[language=C]
// Subscribe to event
EventManager.Name_of_event += Name_of_function;

// Unsubscribe from event
EventManager.Name_of_event -= Name_of_function;
        \end{lstlisting} 
        This component also has the responsibility to handle the laser pointer and the interaction with the controllers. It has a reference to the ``LaserPointer'', a component attached to the ``LaserPointerPlayer'' object, which gives the ``EventManager'' information about the position and orientation of the pointer. Every frame it then uses this information to fire a ray cast from the tip of the laser pointer in the direction it faces. After that it uses the result of the ray cast in order to check whether it is pointing at an interactive object. If it is pointing at an interactive object, it will trigger the hover effect of that object, and if the player is also pressing the trigger button while pointing at this interactive object, it will fire the appropriate event according to the type of interactive object. 
    \end{itemize}
    
%colormanager
    \item \textbf{ColorManager}
    \begin{itemize}
        \item[] ``ColorManager'' is a simple component. All it does is provide a way to define colors in the Unity inspector and expose a static instance of this object. This public static object reference makes the defined colors available anywhere within the scene without having to instantiate an instance of this class or having a reference to one. It is also set up to have a Singleton architecture, meaning that there will only ever be a maximum of one instance of this class at any given time. 
    \end{itemize}
    
%uibutton
    \item \textbf{MenuButton}
    \begin{itemize}
        \item[] ``UIButton'' is an abstract component, and also the parent class of ``MenuButton''. The ``UIButton'' component's function is to provide an animation for pressing it, and also give the possibility to add methods to trigger when it is pressed. Methods can either be added through code, or through the inspector in Unity. To achieve this possibility an event class inheriting from the generic ``UnityEvent\textless T\textgreater'' class was created, internally in the ``UIButton'' class. The ``T'' can be replaced with whatever type you want to pass to this event. In this case:
        \begin{lstlisting}[language=C]
[Serializable]
public class ButtonEvent: UnityEvent<UIButton> { }

public ButtonEvent onButtonPressed;
        \end{lstlisting}
        As we can see, the ``ButtonEvent'' inherits from ``UnityEvent\textless UIButton\textgreater''. This means that methods subscribed to this event will get a ``UIButton'' instance passed to them. This was to ensure that any property or functionality of any sub-classes of ``UIButton'' could be used if needed. The ``UIButton'' class itself, however, contains no useful information to be used by the subscribed method. The ``[Serializable]'' part along with the public ``ButtonEvent'' variable, enables the subscriptions to be done trough the Unity inspector. Calling all the subscribed methods is just a matter of calling the ``Invoke'' method and passing an instance of the required type along with it. like this:
        \begin{lstlisting}[language=C]
// "this" is in this case the UIButton instance
onButtonPressed.Invoke(this);
        \end{lstlisting}
        The ``MenuButton'' component just inherits from ``UIButton'' without altering or extending it in any way. The only reason it exists is to have a component name which indicates what kind of button this is. Other sub-classes of ``UIButton'' are ``ActionButton'' and ``ModeButton''. ``ActionButton'' is the most extended version. It adds functionality for blinking the buttons, as to hint which action to perform next, and it allows for action buttons to be multi-stepped. This means that when an action button which is set up as a multi-step action is pressed, it will enter an ``in progress'' state. And to complete this action, the next selection will be registered as the continuation of this action. ``ModeButton'' has only one addition, which is expose a field which can be used to specify what mode this button trigger.  
        \end{itemize}    
        
%SceneChanger
    \item \textbf{MainSceneChanger}
    \begin{itemize}
        \item[] ``SceneChanger'' is an abstract component, and also the parent class of ``MainSceneChanger''. All it does is provide a method for transitioning to another scene by fading to black and then fading back in to the new scene. To do this it utilizes the ``SteamVR\_LoadLevel.Begin()'' method, provided by the SteamVR plugin, passing the name of the scene to load and the duration of the fade as parameters.
        
        Current sub-classes are: ``MainSceneChanger'', and ``AlgDatSceneChanger''.
    \end{itemize}
    
    \item \textbf{PlayerPositioner}
    \begin{itemize}
        \item[] ``PlayerPositioner'' is a component. Its function is to find the player and position it at the location of the game object this component is attached to, when the scene starts. This means that the ``\_FallbackPlayer'' prefab can be placed anywhere in the scene as long as there a ``PlayerPositioner'' component at the position where the player should be. 
    \end{itemize}

\end{itemize}

\subsubsection{Course Hub for Algorithms and Data Structures}
The course hub for the algorithms and data structures course contains nothing more compared to the overworld scene. The only difference is the sub-class of ``SceneChanger'' used; ``AlgDatSceneChanger'' as opposed to ``MainSceneChanger''.

\subsubsection{Data Structures}
The data structures scene is split into two parts. These are the play mode, and the use case mode. It starts out in the play mode. In this mode the player can choose between the stack and the queue structures, and then get a visual representation of what happens when data is added or removed to the selected structure. The bits common to both parts are:
\begin{itemize}
    \item \textbf{\_FallbackGame}
    \item \textbf{PlayerPositioner}
    \item \textbf{AlgDatSceneChanger}
    \item \textbf{Stage}
    \begin{itemize}
        \item[] The ``Stage'' component exposes methods for switching between the play- and use case modes, and makes sure both these modes are updated with the correct data structure to use(the chosen structure). 
    \end{itemize}
\end{itemize}

In play mode, these are the components working the scene:
\begin{itemize}
    \item \textbf{DStack}
    \begin{itemize}
        \item[] ``DStack'' is a component, and the naming of this is not very representative of what it does. It actually contains all the logic for both the stack and the queue. Originally it was split into two but realizing that they were very similar and that only a boolean value was needed to tell them apart, they were merged, and the ``DStack'' was arbitrarily chosen over the ``DQueue'', and it was never renamed to fit its contents better. Its contents are definitions for animations for both the stack and the queue in regards to adding and removing data, exposes methods for triggering these animations, keeps references to a few positions it uses for animations, and it keeps track of how many data items it holds and can hold. 
    \end{itemize}

    \item \textbf{Play}
    \begin{itemize}
        \item[] ``Play'' is a component, which sets the ``DStack'' in the correct mode, based on the selected data structure. It also makes sure all of the game objects within the play mode is shown when in play mode, and hidden when in use case mode. 
    \end{itemize}

\end{itemize}

The use case mode is more complex, and the following are the major pieces of it:
\begin{itemize}

%RegionGrowingAlgorithm
    \item \textbf{RegionGrowAlgorithm}
    \begin{itemize}
        \item[] The class ``RegionGrowAlgorithm'' generates a list of actions needed to be performed to extract the connected region/pattern in the image based on a seed point within the image. It exposes methods for retrieving these actions and this enables the game to perform the algorithm one step at a time. 
    \end{itemize}

%ImageHandler
    \item \textbf{ImageHandler}
    \begin{itemize}
        \item[] The ``ImageHandler'' component, is responsible for creating a two dimensional array of ``Pixel'' prefabs to represent an image. It also generates a random pattern in this image, and exposes methods for retrieving individual pixels and for selecting them. The ``Pixel'' prefab contains a cube mesh and the ``Pixel'' component, which has a boolean value which controls weather this pixel is black or white. The ``Pixel'' component is one of the components the ``EventHandler'' component looks for when pointing the laser pointer, making pixels selectable/interactive. 
    \end{itemize}
    
%DataStructure
    \item \textbf{DataStructure}
    \begin{itemize}
        \item[] The ``DataStructure'' component does alot of the same as the ``DStack'' component, except it is modified to make the animations fit this part of the scene and with a few additions. 
    \end{itemize}
    
%ImageRep
    \item \textbf{ImageRep}
    \begin{itemize}
        \item[] ``ImageRep'' is a component which generates a representation of the state within the algorithm. It does so by generating a miniature version of the image, but without the pattern. The colors of each pixel represents the state of it, and it can be one of the following: unvisited(default), visited, seed, next, added to stack/queue, or part of patter. It also shows a legend of all the colors and their meaning.
    \end{itemize}
    
%PatternRep
    \item \textbf{PatternRep}
    \begin{itemize}
        \item[] The ``PatterRep'' component is similar to the ``ImageRep'' but in contrast only has two states: part of patter, or not. It shows which pixels are confirmed as part of a region or pattern this far in the algorithm.
    \end{itemize}

%usecase
    \item \textbf{UseCase}
    \begin{itemize}
        \item[] ``UseCase'' is a component, and it does the opposite of the ``Play'' component, by hiding its belonging game objects when in play mode, and showing them when in use case mode. However, it also contains a lot more. It has references to all the parts in this part of the scene, all mentioned above, and it contains the logic for how and when the user can interact with them. 
    \end{itemize}
    
    \item \textbf{}
    \begin{itemize}
        \item[] 
    \end{itemize}
\end{itemize}

\subsubsection{Sorting Algorithms}
The scene containing the sorting algorithms is the most complex of the scenes. 
Here are the interesting bits:
\begin{itemize}
    \item \textbf{SortingManager}
    \begin{itemize}
        \item[] The SortingManager is a component which basically glues everything together, and keeps track of and handles most of the logic for the flow of the game. Basically the big boss commanding all its minions. This class is however dependent on a lot of other smaller components. This class does not know \textit{how} to do things, only \textit{what} to do. Therefore it must have references to the components which actually know \textit{how} to perform these actions or handle specific events. It does not only know \textit{what} to do, it also knows \textit{when this is allowed} to do. In other words, it keeps track of a bunch of states and when it is notified of some interaction happening, it will decide whether or not this is legal and act accordingly. 
        Some references it keeps are :
        \begin{itemize}
            \item SortingAlgorithm - An algorithm which the array should be sorted after
            \item Arrays - Keeps track of all the arrays, and the logic for interacting with them
            \item Comparison - Handles compare actions
            \item TitleManager - Updates the title of the room to the current algorithm
            \item StateManager - Shows the updated state of the algorithms internals
            \item ActionManager - Handles the state of the Action buttons (active or not)
            \item AlgoControlManager - Handles the state of the buttons controlling the algorithm(Demo, prev, and next)
            \item MessagesManager - Shows messages to the player about things that are happening
        \end{itemize}
    \end{itemize}
    
    \item \textbf{Selectable}
    \begin{itemize}
        \item[] ``Selectable'' is an abstract component, which is the parent class for all selectable game objects in this scene except for the buttons. The selectable class contains everything which is common for all these objects, namely that they can be selected, and a few other things. Current sub-classes are the ``ArrayManager'', ``SortingElement'', ``EmptyElement'', and ``PartialArray'' components. These can all be selected and performed actions on. 
    \end{itemize}
    
    \item \textbf{SortingElement}
    \begin{itemize}
        \item[] ``SortingElement'' is both a prefab, and component, which is also attached to this prefab. The prefab contains a cube mesh to represent the array element, text to show which element this is, and a ``SortingElement'' component. This component contains some information about the element which is useful or needed by other components when performing actions on these. It is also as mentioned a sub-class of ``Selectable''.
    \end{itemize}
    
    
    \item \textbf{ArrayManager}
    \begin{itemize}
        \item[] The ``ArrayManager'' component used to be the main component in terms of keeping track of all the ``SortingElement'' game objects, animating their positions and sizes, and expose methods for performing these actions. It is now reduced to only keeping track of its belonging elements, and exposes methods for getting these elements. The new main component for the arrays and elements now, is ``Arrays''.
    \end{itemize}
    
    \item \textbf{Arrays}
    \begin{itemize}
        \item[]``Arrays'' is a component. It handles the logic of how the different actions affect the array(s), and last but not least which array it should perform the action on. In the beginning this component contained all the animations for the arrays and array elements too, but it got really big and I wanted to separate the animations of the different actions with the logic behind them. And hence the ``ElementAnimator'' was born. However, it turned out some of the logic still had to happen within the animation and are therefore not completely separate after all... Nevertheless, it turned out to be helpful in terms of navigating the code, when all the animation logic was in one file, while most of the logic was in the other.

        The ``Arrays'' component is also an evolution from ``ArrayManager''. As I added the merge sort algorithm there was need for more than one array, and therefore the ArrayManager which was doing everything for the original array was not sufficient any more. Hence the ``Arrays'' component was created. It stole the logic from ``ArrayManager'' and added some more to handle several arrays. The ``ArrayManager'' was cut down to only keep track of its elements. 
        So the history went like this:
        \begin{itemize}
            \item ArrayManager
            \item ArrayManager and Arrays
            \item ArrayManager, Arrays and ElementAnimator
        \end{itemize}
    \end{itemize}
    
    \item \textbf{PartialArray and CombinedArray}
    \begin{itemize}
        \item[] The ``PartialArray'' and ``CombinedArray'' components, were added alongside ``Arrays'' during the implementation of merge sort. These contain a reference to a prefab which lets them spawn a new array cube with its own elements. And they are used when splitting an array in two halves, and merging them back together. They contain similar logic to ``ArrayManager''.
    \end{itemize}
    
    \item \textbf{SortingAlgorithm}
    \begin{itemize}
        \item[] ``SortingAlgorithm'' is an abstract class, and defines a set of abstract methods which all sub-classes needs to implement, a set of virtual methods which already has a definition but can be overridden by the sub-classes, and a set of defined methods which will always be the same for all sub-classes. Having this abstract class makes adding new algorithms a lot faster, as a lot of the code for each algorithm would be the same. It also makes the addition of new algorithms less error prone, as the core is already there for each algorithm and only what is specific for the algorithm needs to be added. The abstract methods are the ``GenerateActions'' method, and the ``CheckForFocusChange'' method. ``GenerateActions'' return Void and takes no parameters. It is called in the constructor and the intention of this method is to add all the ``GameActions'' needed to sort the array in the correct order and with the correct indexes to the actions list, as well as add an equal number of states to the states list to let the user peek inside the state of the algorithm. The ``CheckForFocusChange'' method should trigger an event in the ``EventManager'' when the algorithm reaches a state where there is a change in which elements are in focus.
 
    \end{itemize}
    
    \item \textbf{}
    \begin{itemize}
        \item[] 
    \end{itemize}
    
    \item \textbf{}
    \begin{itemize}
        \item[] 
    \end{itemize}
\end{itemize}

\subsubsection{Animations}
A lot of the components explained above mention animations, and that they keep definitions of animations, and ways to trigger them. However, none explain how they are defined and how they work. The animations were coded, and the way the animations were coded was by using coroutines. Couroutines are a part of the ``UnityEngine'' library, and they work by defining a method with a return type of ``IEnumerator'', and should contain the code for doing the animation, i.e move, rotate, and or scale objects. And to start the defined animation one needs to call the ``StartCoroutine'' method and provide the animation function as a parameter to that function. 
Like this:
\begin{lstlisting}[language=C]
IEnumerator SomeAnimation(){
    float duration = .6f;
    float elapsed = 0f;
    float prevTime = Time.time;
    //This while loop will run one iteration of the loop every frame for .6 seconds. This way of writing the coroutine makes the animation's duration independent of the frame-rate
    while(elapsed < duration){
        // Do something every frame here
        ...
        float time = Time.time;
        elapsed += time - prevTime;
        prevTime = time;
        yield return null;
    }
    //Do something at the end of the animation here
    ...
}

...
// Somewere inside a function which should trigger the SomeAnimation animation
StartCoroutine(SomeAnimation());
\end{lstlisting}

The special part about these coroutines is that is executes code until it reaches a ``yield return ...'' statement, then it will stop executing, and then continue from where it left off at the next frame. There is no limit on the amount of ``yield return ...'' statements a coroutine can have. the value after ``yield return'' can be either null or one of several different types for waiting specific times until resuming execution, like a set amount of time in seconds, or at the end of the frame instead of the beginning of the frame, and so on. 

\begin{comment}
\subsubsection{Major Components}
\label{sec:majorComponents}
Now, it is time to explain the most crucial bits of the code. The different pieces will be classified as either one of these: 
\begin{itemize}
    \item A component - A C\# class inheriting from MonoBehaviour.
    \item A prefab - A template for a game object which can be used to spawn predefined game objects at run-time or for speeding up the scene building process.
    \item A class - A C\# class
    \item An interface - A C\# interface
\end{itemize}
\end{comment}

%\subsubsection*{LaserPointerPlayer} \label{subsubsec:LPP} 
%The ``LaserPointerPlayer'' prefab was a modified version of the Player prefab which is shipped out of the box with Steam VR. The Player prefab controls the tracking of the controllers and the HMD. The modified version removed the hand models, and their gesture simulation as this is not needed for the interaction within the game. Instead the hand models were replaced by spheres and attached to the right hand was a laser pointer which is used for interaction. Along with this laser pointer model, a component was added to control this model. 

%\subsubsection*{\_FallbackGame}
%``\_FallbackGame'' is a prefab, and it came to be after a bug found late in the development. Shortly explained, the \hyperref[subsubsec:LPP]{LaserPointerPrefab} prefab object would not be destroyed when changing scenes and there would be multiple of these when more than one scene was used. More on this in the next section. The purpose of this ``\_FallbackGame'' object was to check for a player and if there was none, instantiate one, else, destroy itself. This made it possible to start the game from any scene and still visit other scenes without getting multiple players, and without only having a player in one of the scenes. This was the initial function of this prefab, but since there would only ever be one of these objects, the opportunity arose to move all other components which were required by all scenes into this same object. Kind of making it into the core of the game, and hence the name ``\_FallbackGame'', where ``Fallback'' comes from the part where it checks whether or not it is needed, and ``Game'' coming from the contents being all the stuff required everywhere within the game. The contents are, excluding the player, the ``EventManager'' and the ``ColorManager'' components, as of the time of writing. 

%In the prototype the sorting algorithms are based around a "SortManager", an "EventManager", the "SortingElement", the "ISortingAlgorithm", and the "GameAction". 
\begin{comment}
\subsubsection*{SortingManger}
The SortingManager is a component which basically glues everything together and keeps track handles most of the logic for the flow of the game. Basically the big boss commanding all its minions. This class is however dependent on a lot of other smaller components. This class does not know \textit{how} to do things, only \textit{what} to do. Therefore it must have references to the components which actually know \textit{how} to perform these actions or handle specific events. It does not only know \textit{what} to do, it also knows \textit{when this is allowed} to do. In other words, it keeps track of a bunch of states and when it is notified of some interaction happening, it will decide whether or not this is legal and act accordingly. 
Some references it keeps are :
\begin{itemize}
    \item \hyperref[subsubsec:SortingAlgorithm]{SortingAlgorithm} - An algorithm which the array should be sorted after
    \item \hyperref[subsubsec:Arrays]{Arrays} - Keeps track of all the arrays, and the logic for interacting with them
    \item Comparison - Handles compare actions
    \item TitleManager - Updates the title of the room to the current algorithm
    \item StateManager - Shows the updated state of the algorithms internals
    \item ActionManager - Handles the state of the Action buttons (active or not)
    \item AlgoControlManager - Handles the state of the buttons controlling the algorithm(Demo, prev, and next)
    \item MessagesManager - Shows messages to the player about things that are happening
\end{itemize}

%this needs a rewrite 
\subsubsection*{Arrays and ElementAnimator} \label{subsubsec:Arrays}
"Arrays" and "ElementAnimator" are components used in the sorting algorithms scene. They handle the logic of how the different actions affect the array(s) and their respectful animation definitions, and last but not least which array it should perform the action on. In the beginning they were the same component, but it got really big and I wanted to separate the Animations of the different actions with the logic behind them. And hence they became two. However, it turned out some of the logic still had to happen within the animation and are therefore not completely separate after all... Nevertheless, it turned out to be helpful in terms of navigating the code, when all the animation logic was in one file, while most of the logic was in the other.

The "Arrays" component is also an evolution from "ArrayManager". As I added the merge sort algorithm there was need for more than one array, and therefore the ArrayManager which was doing everything for the original array was not sufficient any more. Hence the "Arrays" component was created. It stole the logic from "ArrayManager" and added some more to handle several arrays. The Arraymanager was cut down to only keep track of its elements. 
So the history went like this:
\begin{itemize}
    \item ArrayManager
    \item ArrayManager and Arrays
    \item ArrayManager, Arrays and ElementAnimator
\end{itemize}


\subsubsection*{EventManager}
The EventManager is a class which contains every event which could happen in the game, and allows other classes or components to subscribe to these events. The events are set up with delegate types for the different types of events, and matching static events using these delegate types. making the events static means that other components does not need a reference to the active EventManager in order to subscribe to events. Then, all a component need to do in order to subscribe to an event is to define a function which takes the same parameters as the delegate type of the event, and append that function to the event. Like this: 
\begin{lstlisting}[language=C]
// Subscribe to event
EventManager.Name_of_event += Name_of_function;

// Unsubscribe from event
EventManager.Name_of_event -= Name_of_function;
\end{lstlisting} 

This component also has the responsibility to handle the laser pointer and the interaction with the controllers. It has a reference to the LaserPointer, a component attached to the \hyperref[subsubsec:LPP]{LaserPointerPlayer} object, which gives the EventManager information about the position and orientation of the pointer. Every frame it then uses this information to fire a ray cast from the tip of the laser pointer in the direction it faces. After that it uses the result of the ray cast in order to check whether it is pointing at an interactive object. If it is pointing at an interactive object, it will trigger the hover effect of that object, and if the player is also pressing the trigger button while pointing at this interactive object, it will fire the appropriate event according to the type of interactive object. 
\end{comment}

%\subsubsection*{SortingAlgorithm} \label{subsubsec:SortingAlgorithm}
%SortingAlgorithm is an abstract class, and defines a set of abstract methods which all sub-classes needs to implement, a set of virtual methods which already has a definition but can be overridden by the sub-classes, and a set of defined methods which will always be the same for all sub-classes. Having this abstract class makes adding new algorithms a lot faster, as a lot of the code for each algorithm would be the same. It also makes the addition of new algorithms less error prone, as the core is already there for each algorithm and only what is specific for the algorithm needs to be added. The abstract methods are the GenerateActions method, and the CheckForFocusChange method. GenerateActions return Void and takes no parameters. It is called in the constructor and the intention of this method is to add all the GameActions needed to sort the array in the correct order and with the correct indexes to the actions list, as well as add an equal number of states to the states list to let the user peek inside the state of the algorithm. The CheckForFocusChange method should send trigger an event in the EventManager when the algorithm reaches a state where there is a change in which elements are in focus.

%\subsubsection*{GameAction}
%GameAction is an abstract class which all actions that a user can perform inherits from. This class contains very little functionality as its reason for existing was to be able to group several different actions under the same type, such that collections could be made. By inheriting from this class one can also quite easily add new actions as well, though they also have to be defined in the SortManager, as these classes are only a representation of an action, and not the action itself. The functionality of the action is decided by the SortManager. In this way, the same action can be handled diffrently in another context, for example for tree traversal algorithms. 


\subsubsection{Bugs and Stumbles}
\todo{Not sure if this part should even be included...}
% the problem of VR player not beeing deleted between scenes, and the solution to that

Improper use of the player prefab from the Steam VR Plugin was the source of one of the bugs found during development of this prototype. It was not found until very late in the process as the scenes were usually only tested individually, and not after transitioning from one to the other. As it turned out the player prefab was not destroyed on scene change, and every scene had one of these player prefabs, which meant there would be several player objects in the scene after the first scene was left. This was not obvious as most of my scenes had the player object in the exact same spot and they were just overlapping each other. However, when transitioning from a scene were this object was in another position than the previous scene, it became very obvious as one could see two laser pointers originating from different spots. The solution to this was to create the  \_FallbackGame prefab, whose functionality is already described in the previous part. After this prefab was constructed, all that was left was just to replace the player prefab in all the scenes and add this prefab instead. Also, the ColorManagers, and EventManagers could be removed from the scenes, as they were included in the fallback prefab. This also made for a better architectural design as any changes to the ColorManager or EventManager would be applied instantly to all scenes by just altering the prefab, keeping colors and settings consistent across the game. 

\subsubsection{Features}
The following features were present in the final version of the prototype:

\subsubsection*{Common to all scenes}
\begin{itemize}
    \item Player - A camera and controller reflecting the physical orientation and position of the player hand his/her hands/controllers. Also has a laser pointer in the right hand for interacting with the world
    \item Show outline on interactive objects when hovering over with the laser pointer
    \item Interact with interactive objects by pressing the trigger button on the controller with the laser pointer while pointing at it
    \item Buttons - Interactive object
    \item Show which buttons are disabled by changing their color, and the color of the outline when hover over
\end{itemize}

\subsubsection*{Over world}
\begin{itemize}
    \item Show available courses as different buttons
    \item Course buttons - Loads the button's course
    \begin{itemize}
        \item Algorithms and Data Structures
    \end{itemize}
\end{itemize}

\subsubsection*{Algorithms and Data Structures}
\textbf{Course hub(entry point):}
\begin{itemize}
    \item Show available topics as buttons
    \item \textbf{Topic buttons} - Loads the button's topic
    \begin{itemize}
        \item Data structures
        \item Sorting algorithms
    \end{itemize}
    \item Back navigation button - Go back to the over world
\end{itemize}

\textbf{Data structures:}
\begin{itemize}
    \item \textbf{Play mode} - Learn how the selected structure works
    \begin{itemize}
        \item Show available structures as buttons
        \item \textbf{Structure buttons} - Go to play mode for selected structure
        \begin{itemize}
            \item Stack
            \item Queue
        \end{itemize}
        \item \textbf{Stack and Queue} - Same play mode
        \begin{itemize}
            \item Show a representation of the data structure
            \item Push/Enqueue button - Animate how the structure handles addition of data
            \item Pop/Dequeue button - Animate how the structure handles removal/retrieval of data
            \item Use case button - Go to use case mode for this structure
            \item Back navigation button - Go back to course hub
            \item Show explanation of each button
            \item Show error message when doing an illegal action containing the error (Overflow/Underflow) and when it occurs
        \end{itemize}
    \end{itemize}
    \item \textbf{Use case mode} - See how the selected structure can be used in a real life scenario
    \begin{itemize}
        \item \textbf{Stack and Queue} - Same use case: Image region growing
        \begin{itemize}
            \item Back navigation button - Go back to play mode for this structure
            \item Show a description of what this use case is when entering the use case mode. Press the button on this panel to hide it and start the algorithm.
            \item Show a black and white image with random pattern. Each pixel is an interactive object.
            \item Pixel(interactive object) - Select (show a border color to indicate that it is selected)
            \item Show a representation of the data structure selected before entering use case mode
            \item Show a representation of the the item from the data structure is currently being read/used ("Next" as it points to the next pixel to check)
            \item Show a representation of the internal state of the algorithm (Pattern so far, visited pixels, which pixel is next, etc...)
            \item Show an explanation of the color coding in the representation of the algorithm
            \item \textbf{Action buttons}
            \begin{itemize}
                \item Push/Enqueue - Animate addition of the selected Pixel to the data structure
                \item Pop/Dequeue - Animate removal of next pixel from the data structure and into the "Next" position. 
                \item Check - Animate the checking(whether it is or is not part of the pattern) of the pixel.
            \end{itemize}
            \item \textbf{Algorithm interaction buttons}
            \begin{itemize}
                \item Demo(toggle button) - Start/stop the demonstration of the algorithm. The demonstration animates how the algorithm works step by step.
                \item Prev - Undo the last step in the algorithm(animate)
                \item Next - Do the next step in the algorithm(animate)
            \end{itemize}
            \item An attempt at doing an action the algorithm does not expect, it will not perform the action and instead hint at the expected action by blinking the required elements and action button a few times. 
            \item Data items popped or dequeued into "Next" will fire a laser at the pixel it represents. To indicate which pixel should be checked next
        \end{itemize}{}
    \end{itemize}
\end{itemize}

\textbf{Sorting Algorithms}
\begin{itemize}
    \item Back navigation button - Go back to course hub
    \item Show available sorting algorithms as buttons
    \item Show the array as a horizontal set of cuboids(array elements) on top of another cuboid(array stand or representation of the memory location). The elements has a height relative to their value: high value means a tall cuboid, and a low value means a short cuboid.  
    \item Both the array and the the elements are interactive
    \item Show a Cube representing the storage (where array elements can be temporarily stored)
    \item Show a representation of the internal state of the algorithm
    \item Show pseudo code for the selected algorithm
    \item \textbf{Sorting algorithm buttons} - Generates a new random array and changes to this algorithm
    \begin{itemize}
        \item Bubble sort
        \item Insertion sort
        \item Quick sort
        \item Merge sort
    \end{itemize}
    \item \textbf{Algorithm interaction buttons}
    \begin{itemize}
        \item Demo(toggle button) - Start/stop demonstrating the algorithm. The demonstration shows the algorithm step by step in the correct order and explains what it is doing.
        \item Prev - Undo the last step in the algorithm(animate)
        \item Next - Do the next step in the algorithm(animate)
        \item New array - Generate a new random array
        \item Restart - Resets the algorithm, and array back to start    
    \end{itemize}
    \item \textbf{Action buttons}
    \begin{itemize}
        \item Compare - Compare the value of two elements
        \item Swap - Swap the values/position of one or two element(s)
        \item Store - Clone one of the elements in the array to the storage
        \item Copy to - Two step action: Copy selected element value into the next selection (limitations: can not copy to storage, use Store instead)
        \item Pivot - Sets the selected element as a pivot
        \item Split - Splits the selected array in two halves
        \item Merge - Start the merge of two arrays
    \end{itemize}
    \item An attempt at doing an action the algorithm does not expect, it will not perform the action and instead hint at the expected action by blinking the required elements and action button a few times. 
    
\end{itemize}



\subsection{Testing}

The testing was performed in mid May, and 9 people participated. As planned, the participants were given a short description of the game and its intentions to give them some context. They were also informed that they may ask questions if they get stuck, but refrain from doing this needlessly. After the explanation they were asked to equip the VR headset hand the controller, and play the game. Then they played until they felt that they had explored everything, or until they felt they had seen enough. Finally they were asked to fill out a google form. And during the last step some of the participants gave some oral feedback as well. This was all according to the plan presented in the method section. 

Let us see who the participants were. According to their answers in the google form, all of them were students, and all but one had previously taken the course ``Algorithms and data structures''. The majority of the participants had tried VR earlier, but were not very experienced with it. The details can be seen in \autoref{fig:student}, \autoref{fig:VRExp} and \autoref{fig:AlgDat}. 

\begin{figure}[H]
\centering
\includegraphics[width=\textwidth]{images/Student.png}
\caption{The share of the participants in the testing currently being students.}
\label{fig:student}
\end{figure}

\begin{figure}[H]
\centering
\includegraphics[width=\textwidth]{images/VRExp.png}
\caption{The amount of experience the participants in the testing had, prior to the testing.}
\label{fig:VRExp}
\end{figure}

\begin{figure}[H]
\centering
\includegraphics[width=\textwidth]{images/AlgDat.png}
\caption{The relation between the participants in the testing and the course ``Algorithms and data structures''.}
\label{fig:AlgDat}
\end{figure}

The results of the SUS gave an average score of 70.3. The average SUS score in general is 68, which means that the usability of this prototype scored above average. The participant which had not taken the course in the prototype gave a score of 85 which is the next highest score given. Only including the participants which had taken the course, made the average score drop to 68.4, which is only slightly above average. This is the almost the opposite result of the testing in the previous project. In that project the lowest score came from the participants not having taken the course, and the average rose when excluding these. 

VR is infamous for its tendency to cause motion sickness in players, and some of the participants said they were experiencing this during the testing. In fact three out of the nine participants said that they felt some degree of motion sickness, which is 33.3\%. Some participants also felt other discomforts such as the game making them feel dumb, or due to other reasons. The details can be seen in \autoref{fig:discomfort}.

\begin{figure}[H]
\centering
\includegraphics[width=\textwidth]{images/VRNegativeEmotion.png}
\caption{The response from the participants in the testing in regards to negative emotions felt during the testing.}
\label{fig:discomfort}
\end{figure}

The thoughts of the participants in regards to using VR for teaching/learning were very positive, and their experience with the prototype was for the majority good. In fact everyone thought that VR in general could be useful for learning! Almost all thought that an application like the prototype also could be useful. The answers can be seen in \autoref{fig:enjoy}, \autoref{fig:useful} and \autoref{fig:generalUseful}.

\begin{figure}[H]
\centering
\includegraphics[width=\textwidth]{images/VREnjoy.png}
\caption{Whether or not the participants enjoyed the experience of playing ``UniVRsity''.}
\label{fig:enjoy}
\end{figure}

\begin{figure}[H]
\centering
\includegraphics[width=\textwidth]{images/VRThisUseful.png}
\caption{Whether or not the participants of the testing thought that ``UniVRsity'' could be useful for teaching/learning.}
\label{fig:useful}
\end{figure}

\begin{figure}[H]
\centering
\includegraphics[width=\textwidth]{images/VRGeneralUseful.png}
\caption{Whether or not the participants of the testing thought VR could be useful for teaching/learning in general.}
\label{fig:generalUseful}
\end{figure}

The general feed back given was that it was not clear that the user could do the steps in the algorithms manually. Most only used the ``Demo'' or ``Do next step'' buttons when playing the game. Some also said that a lot of things happened simultaneously and it was hard to know where to focus, specifically for the use case in the data structures scene.  




\newpage
\section{Discussion}
This chapter will process the information and experience gathered and gained throughout the development and testing, and reflect around them. What was done well? What could have been done better? What have I learned? Does the results make sense? Is the data gathrerd sufficient for reaching a conclusion? Questions like these are what you might find in this chapter. Starting with the process and planning, then moving on to the development, and lastly the testing.

\subsection{Planning and Process}
It would seem as though the previous article was completely forgotten as I have learned next to nothing from it when it comes to planning and process. I redid the same mistakes which I did in the previous project. Too little upfront planning, and too much focus on the development. However, the biweekly meeting with my supervisor forced me to plan a bit ahead, and it gave me a sense of direction and also some sense of urgency. 

\subsection {Development}

\subsection{Testing}
Testing was supposed to have started in late April or early May, but ended up happening mid May. The reason for this was a mixture of things. I was attending a confirmation of my cousin in Stavanger, and due to the SAS-strike I had to travel earlier than needed, and I got home later than planned. During this period I was not able to work too much on the project. This was late in April. Another reason was that I had a realization when I was showing VR to a friend and she wanted to see what I was making. I realized a bunch of tweaks I could make to improve the usability. And some of these tweaks took a lot longer to implement than I had thought. And since quite a few bugs were found in the last projects testing, I thoroughly tested the application in ways it was not meant to be used. This unveiled many bugs which I had not noticed earlier because I, the creator, know how the application is supposed to work and use it thereafter when testing new features. However, some of these bugs were surprisingly hard to squish, but As the testing went smoothly around this time, it seems it was not in vain. 
\begin{comment}
%diskutér resultatene 
Discussion

Planning/Process
The process in this project was handled a bit unprofessionally. I was more eager to develop rather than plan the development. This is not necessarily a bad thing, but the way I handled it could have been much better. The way I did it was that I got an idea of how i wanted the prototype to be, then start to make it. Eventually I would come to a point where I had something which could have been tested, but something about it was not to my liking and I came up with a new design to implement, instead of testing it and basing the new design on feedback from others. This happened at least two times during the project, and cost me a lot of time, with little benefit as I wasted good opportunities to get good feedback.  

Test Results
Now, let us talk a bit about the results from the usability testing. The tests resulted with a SUS score of 67.5 at average. The individual results, however, show that there is a high variance, with the highest score of 85, and the lowest score of 42.5. The thing to take note of here, I think, is that the test subject which gave the lowest score had not taken the course Algorithms and data structures, or any other computer science related subjects, while the other test subjects had taken that course. If we only include the test subjects which had taken this course, the average rises to 75.83, which is a fairly good score. This suggests that the usability of the prototype is linked to the knowledge of the subject, and this is understandable, but might not be a good thing. Because of this, it also suggests that this prototype will perform poorly if used by it self, and not as a supplement to the course. The goal was to make something which could be used as a supplement, but it would, of course, have been nice if it ended up having been self sustainable as well. However, this is only for the usability and does not say anything about the learning/teaching done through this prototype. To test learning is a much more time consuming endeavor. It needs to be done over a longer period of time, and multiple tests, to check for prior understanding, understanding immediately after, and retained knowledge at some point later. This fell outside of the scope of this project.

Research Questions
The first research question introduced int this article was weather VR can be used to help teach complex subjects, and how. Assuming the first part is true, the second part, the "how", then has infinite possibilities. The produced prototype is an attempt at answering this. Weather it accomplishes it is yet to be known. However, the usability test show that people who know something about the subject, find the prototype usable, and might therefor want to use it. This also answers the second question, of how usable the proposed solution is. But then again, usability does not equal engagement(User experience vs Player experience \cite{lazzaro}). And engagement is what one desires in a learning situation. On the other side, bad usability invokes frustration, and not necessarily the good kind of frustration which can transition into mastery. 
\end{comment}
\newpage
\section{Conclusion}
We have reached the end. The project is finished, but what is the conclusion? Well, this is what I think.
\newpage

\bibliographystyle{plain}
\bibliography{references}

\end{document}
